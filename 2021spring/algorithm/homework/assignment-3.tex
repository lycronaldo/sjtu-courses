\documentclass{article}

\usepackage{enumitem} % list item
\usepackage{graphicx} % insert image

% tikz graph
\usepackage{tikz}
\usetikzlibrary{positioning}

\usepackage{times}
\usepackage{a4wide}
\usepackage{latexsym}
\usepackage{mathrsfs}
\usepackage{amsmath}
\usepackage{amssymb}
\usepackage{amsthm}
\usepackage{ifthen}
\usepackage{stmaryrd}
\usepackage{algorithm}
%\usepackage{algorithmic}
\usepackage{algpseudocode}
\usepackage{graphics}
\usepackage{epsfig}
%\usepackage{ramacros}

\addtolength{\textwidth}{20mm}
\addtolength{\oddsidemargin}{-10mm}
\addtolength{\textheight}{18mm}
\addtolength{\topmargin}{-9mm}

\newcounter{exercise}
\setcounter{exercise}{0}
\newcommand{\exercise}{
        \addtocounter{exercise}{1}
        \vspace{0.2in}
        \noindent
        {\bf \theexercise .}
        }

\newcommand{\REMARK}[1]{}


\newcommand{\NEWPART}{\vspace{.1in}
              \noindent}

\newcommand{\<}{
    \langle}

\renewcommand{\>}{
    \rangle}

\newcommand{\ceil}[1]{\left\lceil #1 \right\rceil}

\pagestyle{plain} \pagenumbering{arabic}


\title{{\bf Assignment 3}}
\author{Xian Jianbang - 120037910056}
\date{2021/05/09}

\begin{document}
\maketitle

%{\large \noindent
%\begin{tabular}{lcl}
%{Jan 17 2007}.
%\end{tabular}
%}

{\large





\begin{exercise}
Prove or disprove the following statement. If all capacities in a network are distinct,
 then there exists a unique flow function that gives the maximum flow.
\end{exercise}

\bigskip \noindent
\textbf{Solution}

The proposition is \textbf{false}. In the following graph, all capacities are distinct, but there are 2 max-flow, which are $s \rightarrow a \rightarrow c \rightarrow t$ and $s \rightarrow b \rightarrow c \rightarrow t$ .

\begin{figure}[htp]
    \centering
    \includegraphics[width=6.5cm]{images/A3Q1.png}
    \caption{A graph with two max-flow}
    % \label{fig:galaxy}
\end{figure}


\begin{exercise}
An edge of a flow network is called \textbf{critical} if decreasing the capacity of this edge results in a decrease in the maximum flow value. Present an efficient algorithm that, given an $s$-$t$ network $G$ finds any critical edge in a network(assuming one exists).
\end{exercise}

\bigskip \noindent
\textbf{Solution}

Given an $s-t$ flow network $G$, any edge in the minimum-cut is critical. 

To find such edges, firstly compute max-flow algorithm (Ford-Fulkerson Algorithm) on $G$, then we have 2 methods to find the min-cut:

\begin{itemize}
    \item In the residual network $G_f$, find all vertices that $s$ can reach, as set $A$ (including vertex $s$). Let $B = V-A$, any edge starts from $A$ and ends at $B$ is a critical edge. This method can be done in $O(V)$ time (just do DFS starting from $s$).
    \item In the residual network $G_f$, any edge with $f(e) = c(e)$ is a critical edge. This method can be done in $O(E)$ time (just check all the edges in $G_f$).
\end{itemize}

\newpage

\begin{exercise}
Let $G=(V,E)$ be an undirected weighted graph with two distinguished vertices $s,t\in V$. Give an efficient algorithm to find a minimum weight cut that separates $s$ from $t$.
\end{exercise}

\bigskip \noindent
\textbf{Solution}

We can transform the undirected graph $G$ into a directed graph $G'$ through the follwoing steps:
\begin{itemize}
    \item For any edge $(s, v)$ connected to $s$ in $G$, construct directed edge $s \rightarrow v$ in $G'$.
    \item For any edge $(u, t)$ connected to $t$ in $G$, construct directed edge $u \rightarrow t$ in $G'$.
    \item For any other edge $(u, v)$ in $G$ (which is connected to neither $s$ or $t$), construct 2 edges $u \rightarrow v$ and $v \rightarrow u$ with the same weight in $G'$.
\end{itemize}

Then we can perform max-flow algorithm over $G'$, then we can also get a min-cut of $G'$, which is also a min-cut of $G$. 

\begin{exercise}
You are given a matrix with fractional elements between $0$ and $1$. The sum of all numbers in each row and in each column is integer. Prove that we can always round each element to $0$ or $1$ so that the sum of each row and each column remains unchanged and design a polynomial time algorithm to find such a rounding result.
\end{exercise}

\bigskip \noindent
\textbf{Solution}

Suppose the size of matrix is $n \times m$.

We can reduce this problem into max-flow problem, each row or each column in the matrix is a vertex in the graph. Let:
\begin{itemize}
    \item $r_i$ be the sum of row $i$,
    \item $c_j$ be the sum of column $j$,
    \item vertex $x_i$ be the row $i$, and
    \item vertex $y_j$ be the column $j$.
\end{itemize}

Then we can construct a graph $G=(V,E)$, in which:
\begin{itemize}
    \item $V$ includes a source vertex $s$, a sink vertex $t$, and all the vertices $x_i$ and $y_j$ mentioned above, thus $|V| = m+n+2$.
    \item As for $E$, construct edges from $s$ to each $x_i$ (whose capacity is $r_i$), and construct edges from each $y_j$ to $t$ (whose capacity is $c_j$). Then construct edges from each $x_i$ to each $y_j$, whose capacities are all 1. Thus, $|E| = mn+m+n$.
\end{itemize}

For example, if size of the matrix is $4 \times 4$, then we have vertices $x_1 - x_4$ represent 4 rows, and vertices $y_1 - y_4$ to represent 4 columns, as shown in the figure-2 below.

\newpage

\begin{figure}[htp]
    \centering
    \includegraphics[width=9cm]{images/A3Q4.png}
    \caption{Transformation between network and matrix}
    % \label{fig:galaxy}
\end{figure}

Let $flow(x_i, y_j)$ represent the flow value on edge $(x_i, y_j)$ (either 0 or 1), and let $value(i, j)$ represent the value of $matrix[i, j]$ (either 0 or 1, too). We can find that $flow(x_i, y_j)$ and $value(i,j)$ have the same meaning. As the following 2 equations always hold at the same time:
$$
\begin{aligned}
  &\forall{i \in [1, n]}, & \sum_{\forall{j \in [1, m]}}{flow(x_i, y_j)} &= r_i &= \sum_{\forall{j \in [1, m]}}{value(i,j)} \\
  &\forall{j \in [1, m]}, & \sum_{\forall{i \in [1, n]}}{flow(x_i, y_j)} &= c_j &= \sum_{\forall{i \in [1, n]}}{value(i,j)}
\end{aligned}
$$

In the max-flow of $G$, if $flow(x_i, y_j)$ is 1, then $matrix[i, j]$ should be 1, otherwise 0.

We can say the origin proposition holds \textbf{iff} network flow $G$ has a max-flow with value $\sum_{i=1}^{n}{r_i}$.

\begin{exercise}
Suppose that, in addition to edge capacities, a flow network has \textbf{vertex capacities}.
That is each vertex has a limit on how much flow can pass though. Show how to transform a flow network $G=(V,E)$ with vertex capacities into an equivalent flow network $G'=(V',E')$ without vertex capacities, such that a maximum flow in $G'$ has the same value as a maximum flow in $G$. How many vertices and edges does $G'$ have?
\end{exercise}

\bigskip \noindent \textbf{Solution}

We can do the following transformations for $\forall{v \in V - \{s, t\}}$:
\begin{itemize}
    \item Split $v$ into 2 vertices $v_1$ and $v_2$, and construct edge $v_1 \rightarrow v_2$ in $G'$, whose capacity is the capacity of $v$.
    \item For each edge $x \rightarrow v$ ends with $v$, construct edge $x \rightarrow v_1$ in $G'$.
    \item For each edge $v \rightarrow x$ starts with $v$, construct edge $v_2 \rightarrow x$ in $G'$.
\end{itemize}

After these transformations, $G'$ is an equivalent flow network of $G$.

The number of vertices in $G'$ is $|V'| = 2(|V|-2) + 2$, and the number of edges in $G'$ is $|E'| = |E| + |V| - 2$.

\newpage

\begin{exercise}
Consider a bipartite graph $G=(X\cup Y,E)$ with parts $X$ and $Y$. Each part contains $2k$ vertices (i.e. $|X|=|Y|=2k$). Suppose that $deg(u)\geq k$ for every $u\in X\cup Y$. Prove that $G$ has a perfect matching.
\end{exercise}

\bigskip \noindent \textbf{Solution}

Let $S$ be a subset of nodes, and $N(S)$ be the set of nodes that are adjacent to nodes in $S$.

\smallskip

According to \textbf{Hall's Theorem}, we need to prove that $|N(S)| \ge |S|$ for all subsets $S \subseteq X$.

\smallskip

We will divide all the subsets $S$ of $X$ into 2 parts: $|S| \le k$ and $|S| > k$.
\begin{itemize}
    \item If $|S| \le k$, since $deg(u) \ge k$ for every $u \in X$, $|N(S)| \ge |S|$ must hold in such case.
    \item If $|S| > k$, we prove $|N(S)| \ge |S|$ by contradiction.
    \begin{enumerate}[label=\arabic*)]
        \item Suppose that $|N(S)| < |S|$ .
        \item Because of $|X| = 2k$, we have $|X - S| < k$.
        \item And we can find that $N(X-S)$ is equivalent to $Y - N(S)$. That is to say, $\forall{v \in Y - N(S)}$, $v$ can only be adjacent to vertices in $X-S$. Since $|X - S| < k$, we have $deg(v) < k$ for any $v \in Y - N(S)$, which is in contradiction with $deg(u) \ge k$ for every $u \in X \cup Y$.
        \item Hence we have $|N(S)| \ge |S|$.
    \end{enumerate}
\end{itemize}

In conclusion, $|N(S)| \ge |S|$ always holds. According to Hall's Theorem, $G$ has a perfect matching.


\begin{exercise}
You are designing a experiment in which you want to measure certain properties $p_1,\ldots ,p_n$ of a yeast culture. You have a set of tools $t_1,\ldots ,t_m$ that can each measure a subset $S_i$ of the properties. For example, tool $t_i$ measures $S_i$ may equal $\{p_7,p_8\}$. To be sure that your results are not due to noise or other artifact, you must measure every property at least $k$ times using $k$ different tools.
\begin{itemize}
\item Give a polynomial-time algorithm that decides whether the tools you have are sufficient to measure the desired properties the desired number of times.
\item Suppose each tool $t_i$ comes from manufacturer $M_i$ and we have the additional constraint that the tools to test any property $p_i$ can't all come from the same manufacturer. Give a polynomial-time algorithm to solve this problem.
\end{itemize}
\end{exercise}

\bigskip \noindent \textbf{Solution - 1} \bigskip

This problem can be reduced into \textbf{Circulation with Lower Bound Problem}, as shown in figure-3.

\begin{itemize}
    \item Add source node $s$ and sink node $t$.
    \item For each tool $t_i$, construct edge $s \rightarrow t_i$ with capacity $\infty$.
    \item If tool $t_i$ can measure properties $S_i = \{p_{j_1}, \dots, p_{j_k}\}$, then construct edge from $t_i$ to every $p_{j_1}, \dots, p_{j_k}$, whose capacity is $1$.
    \item For each property $p_j$, construct edge $p_j \rightarrow t$ with capacity $\infty$, but the flow value on these edges must be at least $k$ (i.e. $k \le f(p_j \rightarrow t) \le \infty$).
\end{itemize}

In such case, the origin problem has a solution \textbf{if and only if} there is a valid \textbf{circulation with lower bounds} in the transformed network $G$.

And we can see that, for each augmenting path in $G_f$, its bottleneck is 1, hence the max-flow algorithm can be done in $O(|V||E|)$ time.

\newpage

\begin{figure}[htp]
    \centering
    \includegraphics[width=9cm]{images/A3Q7.png}
    \caption{Transformed network flow $G$}
    % \label{fig:galaxy}
\end{figure}


\bigskip \noindent \textbf{Solution - 2} \bigskip

In this problem, we need to add vertices to represent the relationship between manufacturers and properties.

As shown in figure-4:
\begin{itemize}
    \item We add vertices ${M_i}{p_j}$, which means that tools from manufacturer $M_i$ can measure property $p_j$.
    \item For each tool $t_i$, if it is produced by $M_k$, and could be used to measure $p_j$, then we will:
    \begin{enumerate}[label=\arabic*)]
        \item construct edge $t_i \rightarrow {M_k}{p_j}$ with capacity 1. 
        \item let the capacity of edge ${M_k}{p_j}$ be $k-1$.
    \end{enumerate}
    \item The edges $s \rightarrow t_i$ and $p_j \rightarrow t$ are same as the first problem.
    \begin{enumerate}[label=\arabic*)]
        \item Edge $s \rightarrow t_i$ has $\infty$ capacity.
        \item Edge  $p_j \rightarrow t$ also has $\infty$ capacity, but has a lower bound $k$ of flow value (i.e. $k \le f_e$).
    \end{enumerate}
\end{itemize}


We can see that, for each property $p_j$, it can only be measured $k-1$ times by one tool at most. And this implies that property $p_j$ should be measured by at least 2 tools.

The origin problem has a solution \textbf{if and only if} there exists a valid \textbf{circulation with lower bounds} in $G$.

Same as the first problem, the max-flow here can be done in polynomial time.

\begin{figure}[htp]
    \centering
    \includegraphics[width=12cm]{images/A3Q7-2.png}
    \caption{Transformed network flow $G$}
    % \label{fig:galaxy}
\end{figure}

\newpage

\begin{exercise}
Consider a flow network $G=(V,E)$ with positive edge capacities $\{c(e)\}$. Let $f:E\rightarrow \mathbb{R}_{\geq 0}$ be a maximum flow in $G$, and $G_f$ be the residual graph. Denote by $S$ the set of nodes reachable from $s$ in $G_f$ and by $T$ the set of nodes from which $t$ is reachable in $G_f$.  Prove that $V=S\cup T$ if and only if $G$ has a \textbf{unique} $s$-$t$ minimum cut.
\end{exercise}

\bigskip \noindent \textbf{Solution} \bigskip

\noindent \textit{Proof} $\Leftarrow$
\begin{itemize}
    \item Suppose $V \ne S \cup T$, there must exist vertices (at least one) which are not reachable from $s$ and can not reach any vertex in $T$.
    \item Since all such vertices can not reach any vertex in $T$, we put them into $S$, and we got $S'$. Then $\forall{x \in S'}$, $x$ still can not reach any vertex in $T$. Therefore, $(S', T)$ is also a minimum cut of $G$, which is in contradiction with "$G$ has a \textbf{unique} $s$-$t$ minimum cut".
    \item Hence we have $V = S \cup T$.
\end{itemize}


\bigskip

\noindent \textit{Proof} $\Rightarrow$
\begin{itemize}
    \item When $V = S \cup T$, $(S, T)$ is obviously a minimum cut of $G$. Let the corresponding maximum flow be $f$.
    \item Suppose there is another minimum cut $(S', T')$, and its corresponding maximum flow is $f'$.
\end{itemize}

}
\end{document}


